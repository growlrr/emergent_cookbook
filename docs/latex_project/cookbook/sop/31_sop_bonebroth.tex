% LaTeX fragment — Bone Broth & Skeletal Mince SOP (final)
\section*{SOP: Bone Broth Rendering and Skeletal Mince Preparation for Catpro SKU}

\subsection*{1. Chicken Frame Line}
\textbf{Objective:} Produce standardized calcium- and collagen-bearing broth and skeletal mince for chicken meat SKUs.

\begin{enumerate}
  \item \textbf{Step 1 -  Input Materials:}
    \item Total Time needed 10-15 min 
    \begin{itemize} 
     \item  Log start time in timesheet. 
      \item 2.5\,kg cleaned chicken skeletal frames with 30\% chicken neck and de-clawed feet. Use high velocity hot jet spray to clean off grit. 
      \item 8---10\,kg potable water dependinding on surface area of kettle. Target 4.10Kg rendered bone broth
      \item 0.2\,\% acetic acid (2\,mL/L water)
      \item Log end time in timesheet
    \end{itemize}

  \item \textbf{Step 2- Rendering:}
   \item Total Time needed 12-13 hours with 12 hours of continuous broth rendering time. 
    \begin{enumerate}
      \item Log start time in timesheet. Ensure no more than 10 minutes have passed between end time of Step 1 and start time of Step 2 .
      \item Charge kettle with water and acetic acid; add frames (1:4\,w/v ratio) or as per SOP.
      \item Maintain simmer at 90–95\,$^{\circ}$C for 12\,h; compensate evaporation if needed.
      \item Skim fats and solids every 2\,h.
      \item Reduce to approximately 4.50\,kg$\pm$ 0.4kg finished broth. 
      \item Strain through 1\,mm mesh; 
      \item Lable with date, time and batch number. Refridgerate the broth and use within 12 hours. 
      \item Retain cooked bones and soft matrix for mincing. Refridgerate if needed and use within 12 hours. Lable with date, time and batch number.
      \item Log end time in timesheet. Maintain coldchain logs. Use within 12 hours if refridgerated or immediately if in process. 
    \end{enumerate}

  \item \textbf{Step 3 - Bone Residue Handling and Frame Mince Production:}
   \item Total Time needed 30-40 min 
    \begin{enumerate}
      \item Log start time in timesheet.
      \item Rinse cooked bone residue with warm potable water to remove surface grit and fines.
      \item Remove any dense cortical fragments ($>$10\,mm).
      \item Pass residue twice through 0.3\,mm grinding plate; sieve twice through 0.5\,mm mesh.
      \item Collect fine mince (target yield $\approx$1.65\,kg) for 10kg production run. 
      \item Store $\leq$4\,$^{\circ}$C or freeze at –18\,$^{\circ}$C until use. Lable with date, time and batch number.
      \item Log end time in timesheet.  Maintain coldchain  logs. Use within 12 hours if refridgerated or immediately if in process.
    \end{enumerate}
\end{enumerate}

\subsection*{2. Goat Frame Line}
\textbf{Objective:} Produce lamb skeletal broth and mince with inherent calcium and collagen content for goat meat SKUs.

\begin{enumerate}
  \item \textbf{Step 1 - Input Materials:}
    \item Total Time needed 10-15 min 
    \begin{itemize}
     \item  Log start time in timesheet. 
      \item 3.0\,kg goat skeletal frames with 30\% cleaned, skined, de-hooved goat trotters. Use high velocity hot jet spray to clean off grit. 
      \item 8---10\,kg potable water dependinding on surface area of kettle. Target 4.10Kg rendered bone brotha
      \item 0.2\,\% acetic acid (2\,mL/L water)
      \item Log end time in timesheet
    \end{itemize}

  \item \textbf{Step 2 - Rendering:}
  \item Total Time needed 12-13 hours with 12 hours of continuous broth rendering time.
    \begin{enumerate}
      \item Log start time in timesheet. Ensure no more than 10 minutes have passed between end time of Step 1 and start time of Step 2 .
      \item Charge kettle with water and acetic acid; add bones (1:4\,w/v ratio) or as per SOP.
      \item Simmer 12\,h at 90–95\,$^{\circ}$C; allow slow evaporation to 4.50\,kg $\pm$ 0.4kg broth.
      \item Skim surface fats periodically; strain  twice through 1\,mm mesh . 
      \item Lable with date, time and batch number. Refridgerate the broth and use within 12 hours. 
      \item Retain cooked bones and soft matrix for mincing. Refriderate if needed and use within 12 hours. Lable with date, time and batch number.
      \item Log end time in timesheet. Maintain coldchain  logs. Use within 12 hours if refridgerated or immediately if in process.
    \end{enumerate}

  \item \textbf{Step 3- Bone Residue Handling and Frame Mince Production:}
  \item Total Time needed 30-40 min 
    \begin{enumerate}
     \item Log start time in timesheet.
      \item Rinse residue; discard sharp or metallic fragments.
      \item Double-grind through 0.3\,mm plate; sieve twice through 0.5\,mm mesh.
      \item Target yield $\approx$2.20\,kg fine mince.
      \item Store at $\leq$4\,$^{\circ}$C or frozen (–18\,$^{\circ}$C). Lable with date, time and batch number.
      \item Log end time in timesheet.  Maintain coldchain logs. Use within 12 hours if refridgerated or immediately if in process.
    \end{enumerate}
\end{enumerate}

\subsection*{3. QC and Verification}
\begin{itemize}
  \item Composite samples (100\,g) from each batch to undergo Ca and P assay (ICP or pooled lab test).
  \item Target calcium 400---800\,mg/100\,g; phosphorus 300--400\,mg/100\,g.
  \item Record batch lot numbers; link broth and mince to QC log.
\end{itemize}
