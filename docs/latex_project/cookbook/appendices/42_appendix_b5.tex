\subsection*{Appendix B5 --- B, D, E Complex Strategy}

\textbf{Background:}  
Alongside Vitamin A, B1, choline, and taurine, three additional categories are considered in formulation to ensure both regulatory compliance and biological robustness:  
\begin{itemize}[leftmargin=1.2em]
  \item Water-soluble B-complex vitamins (labile, heat-sensitive).  
  \item Vitamin D (essential but tightly regulated due to tox risk).  
  \item Vitamin E (antioxidant, included both as functional preservative and essential nutrient).  
\end{itemize}

\textbf{Locked Premix Baselines:}  
\begin{itemize}[leftmargin=1.2em]
  \item \textbf{B-complex} ---  
    Thiamine (B1) 0.19 g, Riboflavin (B2) 0.05 g, Niacin (B3) 0.25 g, Pyridoxine (B6) 0.05 g, Folate, Pantothenate, Biotin, B12 (trace).  
    $\rightarrow$ Each pouch (100 kcal) delivers values $\geq$150--200\% of AAFCO/FEDIAF floors, accounting for 20--40\% retort loss.  
  \item \textbf{Vitamin D} ---  
    Locked in premix at negligible mass (0.00025 g/100 g cut), equating to ~100 IU/pouch.  
    $\rightarrow$ Floors: Cat = 62.5 IU/1000 kcal; Dog = 62.5 IU/1000 kcal.  
    $\rightarrow$ Growlrr provides ~1000 IU/1000 kcal baseline, plus natural sardine contribution.  
  \item \textbf{Vitamin E} ---  
    Mixed tocopherols 1.34 g/100 g premix.  
    $\rightarrow$ Each pouch delivers ~200--250 IU/1000 kcal.  
    $\rightarrow$ Floors: Cat = 9.5 IU/1000 kcal; Dog = 9.8 IU/1000 kcal.  
    $\rightarrow$ Overages act both as antioxidant preservative and essential micronutrient source.  
\end{itemize}

\textbf{Natural Organ Contribution:}  
- Egg yolk: B2, biotin, folate; modest Vit D.  
- Sardine: major Vit D source (~200--300 IU/100 g raw).  
- Organs (heart, liver, kidney): Niacin, B6, folate.  
- Pumpkin: trace folate + carotenoids (non-essential but functional).  

\textbf{Compliance Statement:}  
- All three categories exceed AAFCO/FEDIAF minima per 1000 kcal across pouches and weekly rotation.  
- Vitamin D: conservative baseline + sardine ensures floor without risk of tox; internal trigger if >2500 IU/1000 kcal.  
- Vitamin E: delivered at 20--25$\times$ floor, justified as preservative + safe margin; no tox risk in current dosing.  
- B-complex: premix overages ensure post-retort floors are always met; natural organ foods contribute further but are not relied upon for baseline compliance.  

\textbf{SOP Note:}  
- Premix dosing must not be altered; each cut (100 g/10 kg run) is calibrated for compliance.  
- B-vitamin assays post-retort are recommended on pilot lots (esp. thiamine).  
- Vit D audit required if sardine inclusion fluctuates by >10\% of spec.  
- Vit E level also logged as preservative effectiveness (peroxide value QC).  
