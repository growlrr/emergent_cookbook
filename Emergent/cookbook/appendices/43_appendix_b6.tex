\subsection*{Appendix B6 --- Iodine Strategy}

\textbf{Background:}  
Iodine is essential for thyroid hormone synthesis. Both cats and dogs have relatively narrow safe windows: deficiency leads to goiter, developmental stunting, and hypothyroidism; chronic excess can cause hyperthyroidism or thyroiditis.  

\textbf{Regulatory Standards (per 1000 kcal):}  
\begin{itemize}[leftmargin=1.2em]
  \item AAFCO/FEDIAF floor: 0.35 mg  
  \item AAFCO/FEDIAF ceiling: 9.0 mg  
\end{itemize}

\textbf{Locked Premix Baseline:}  
\begin{itemize}[leftmargin=1.2em]
  \item CatPro: 0.015 g iodine / 100 g premix (100 g per 10 kg run $\rightarrow$ 1 g/pouch).  
  \item DogPro: aligned baseline at equivalent dosing.  
  \item Per pouch: ~0.5--0.6 mg iodine/1000 kcal $\rightarrow$ comfortably above the floor, <10\% of ceiling.  
\end{itemize}

\textbf{Natural Organ Contribution:}  
- Poultry heart, gizzard, liver, kidney, lamb muscle/kidney: negligible iodine.  
- Sardine (marine fish): contains iodine, but variable by catch and region (10--30 $\mu$g/g).  
- Curd/yogurt: trace iodine depending on dairy source.  
$\rightarrow$ All natural sources contribute <0.1 mg iodine/1000 kcal, not relied upon for compliance.  

\textbf{Compliance Statement:}  
- Growlrr meets iodine compliance by premix alone ($\geq$0.5 mg/1000 kcal baseline).  
- Natural contributions act as buffer, not baseline.  
- Ceiling is not approached: even with sardine inclusion, totals remain <1.0 mg/1000 kcal.  

\textbf{SOP Note:}  
- Premix iodine spec (0.015 g/100 g cut) must be verified by COA each lot.  
- Salt or seaweed supplements \textbf{must not} be introduced in formulation to avoid ceiling breach.  
- Sardine sourcing should be stable, but iodine is \emph{not} treated as a control variable; premix alone ensures compliance.  
- Routine assay: ICP-MS iodine spot-checks recommended on pilot batches.  
