% =====================================================================
% Growlrr Foods -- SOP 30: Retort Line Floor Operations
% Rev H.3.1.0 - Final production-ready version
% Date: 21 October 2025
% =====================================================================

\section*{SOP 30-A --- Retort Line Floor Operations For CatCore SKUs}
\label{sop:retort_line}

\textbf{Batch Size:} 10 kg run = 100 pouches $\times$ 100.00 g declared weight, target 100.5g $/pm$ 0.5g \\
\textbf{Premix Dosing:} Color-coded QC verification required. CatCore system uses RED stickers for Premix. Refer to SOP 30-B for DogCore SKU.\\
\textbf{Target Calories:} CatCore 100-110 kcal/100g \\
\textbf{Total Production Time:} Approximately 7-8 hours (single production day) with 2 work benches for parallel processing to optimise time, pathogen load and flavor profile. 

\vspace{1em}
\noindent\textbf{Reference:} For detailed QC tolerances, assay protocols, and audit trails, see the detailed SOP binder. This document is for line cooks and supervisors on the shop floor.

\vspace{1em}

% ================================================================
% MASS BALANCE SUMMARY TABLE
% ================================================================

\subsection*{Mass Balance Summary (per 100 pouches)}

\begin{table}[h]
\centering
\caption{Mass Balance for 10 kg Production Run (100 pouches)}
\label{tab:mass_balance}
\begin{tabular}{@{}lrrl@{}}
\toprule
\textbf{Component} & \textbf{Per Pouch (g)} & \textbf{Per Batch (kg)} & \textbf{Notes} \\
\midrule
\multicolumn{4}{l}{\textit{SOLID PHASE}} \\
Raw solids (per SKU BOM) & 65.0 & 6.500 & As specified in SKU formulation \\
Water absorption during blanch & +2.0 & +0.200 & Approximately 3\% weight gain \\
Blanched solids (net) & 67.0 & 6.700 & Ready for marination \\
\midrule
\multicolumn{4}{l}{\textit{TOTAL LIQUID PHASE}} \\
Bone broth (gelatinous base) & 40.0 & 4.000 & From initial 4.10 kg total, set aside 0.10Kg for evaporation loss and dosing errors. \\
Blanch Water recovery & 2.5 & 0.250 & Recovered from blanching \\
Liquid Palatant & 1.0 & 0.10 & Addition of Liquid Palatant - Antioxidant A/B in  Step 1 \\
Broth Liquid Total  & 43.5 & 4.350 & Split into TWO BROTH SYSTEM \\
\midrule
\multicolumn{4}{l}{\textit{LIQUID PHASE --- Broth A (Premix carrier)}} \\
Bone broth (base) & 10.0 & 1.000 & From initial 4.10 kg total \\
CatPro premix & 1.0 & 0.100 & Color-coded packet \\
Blanch water recovery & 2.5 & 0.250 & Recovered from blanching \\
Broth A subtotal & 12.5 & 1.250 & For marination \\
\midrule
\multicolumn{4}{l}{\textit{LIQUID PHASE --- Broth B (Gel system)}} \\
Bone broth (base) & 31.0 & 3.100 & From initial 4.10 kg total \\
Alginate-Ca premix & 1.0 & 0.100 & Added just before filling \\
Broth B subtotal & 32.0 & 3.200 & Gel injection phase \\
\midrule
\textbf{Phase 1 Fill (Solids + Broth A)} & \textbf{80.5} & \textbf{8.050} & First fill (scale tared) \\
\textbf{Phase 2 Fill (Broth B + Gel)} & \textbf{32.0} & \textbf{3.300} & Second fill (injected) \\
\textbf{PRE-RETORT TOTAL (food only)} & \textbf{112.5} & \textbf{11.250} & Target before sealing \\
\midrule
\multicolumn{4}{l}{\textit{Palatant-Antioxidant A/B (distributed in broth)} --- Included in broth totals} \\
\midrule
\multicolumn{4}{l}{\textit{RETORT PROCESS}} \\
Moisture loss during retort ($-$10\%) & $-$11.25 & $-$1.125 & Water evaporates \\
\textbf{POST-RETORT FOOD WEIGHT} & \textbf{101.0} & \textbf{10.100} & Actual food content \\
\midrule
\multicolumn{4}{l}{\textit{FINAL PRODUCT}} \\
Declared food weight & 100.0 & 10.000 & Label claim \\
Pouch tare weight & +4.0 & +0.400 & Empty pouch material \\
\textbf{Final packaged weight (QC target)} & \textbf{104---105} & \textbf{10.400---10.500} & Post-retort scale reading \\
\bottomrule
\end{tabular}
\end{table}

\clearpage

% ================================================================
% LINE SUPERVISOR CHECKLIST
% ================================================================

\subsection*{Line Supervisor Pre-Production Checklist}

\textbf{Complete this checklist before starting production. All items must be checked and signed off.}

\begin{itemize}[leftmargin=1.5em]
\item Prepare Timesheet. Timesheet should have Start time, End time, Target Duration and Actual Duration for each step, Sign off to next step. 
\item[$\Box$] Confirm premix Certificates of Analysis (COAs) are current and QC-approved
\item[$\Box$] Verify batch weights for all premixes, solids, and broth match Bill of Materials (BOM)
\item[$\Box$] Assemble pre-production items:
  \begin{itemize}
  \item Retort pouches (100 units QC checked + 10 spares). 
  \item Labels and batch markers
  \item Calibrated pH meters (QC sign-off required)
  \item Calibrated scales (QC sign-off required)
  \item Calibrated thermometers and sensors (QC sign-off required)
  \end{itemize}
\item[$\Box$] Confirm color-coded premix packets match production schedule:
  \begin{itemize}
  \item \textbf{RED sticker} = CatPro (all CatCore SKUs)
  \item Verify Palatant-Antioxidant variant: A (Standard - RED STICKER) or B (Sardine SKU- BLUE STICKER)
  \end{itemize}
\item[$\Box$] Prepare production log forms and batch record sheets
\item[$\Box$] Obtain all equipment calibration sign-offs from QC before production start
\item Assemble all empty pouches in filling line prior to handling food materials to ensure short handling time. This ensures control of pathogen load and flavor profile. 
\end{itemize}

\vspace{1em}
\noindent\textbf{Supervisor Sign-Off:} \rule{5cm}{0.4pt} \hspace{1cm} \textbf{Date:} \rule{3cm}{0.4pt}

\vspace{1em}
\noindent\textbf{QC Sign-Off:} \rule{5cm}{0.4pt} \hspace{1cm} \textbf{Date:} \rule{3cm}{0.4pt}

% ================================================================
% STEP 1: LIQUID PHASE PREPARATION
% ================================================================

\subsection*{Step 1: Liquid Phase Preparation}

\textbf{Estimated Time:} 15---20 minutes (operator active) \\
\textbf{Personnel Required:} 1 operator \\
\textbf{Equipment Required:} Heating vessel, paddle mixer, calibrated pH meter, thermometer

\vspace{0.5em}
\textbf{Day before production:}
\begin{enumerate}[leftmargin=1.5em]
\item Render bone broth overnight (minimum 12 hours) and skim if needed. Sieve in fine mesh to remove grit. 
\item Chill broth to 4~$^\circ$C . Maintain Cold Chain Logs. Discard if more than 24 hours. 
\item Store covered until production day
\end{enumerate}

\textbf{On retort day:}
\begin{enumerate}[resume,leftmargin=1.5em]
\textbf{Log Start Time in Timesheet:} \_\_\_\_\_ 
\item Measure \textbf{4.10 kg bone broth} and transfer 4.00 Kg to heating vessel. Set aside 0.10Kg for evaporation loss and ops error.
\item Slowly warm 4.00 Kg Broth to 25~$^\circ$C while gently stirring (3---5 min)
\item \textbf{pH Check \#1:}
  \begin{itemize}
  \item Target: pH 6.4---6.5 (5 min)
  \item Use calibrated pH meter
  \item If outside range, adjust with food-grade acid/base (5 min)
  \item \textbf{Obtain QC sign-off before proceeding}
  \end{itemize}
\item Continue warming to 40~$^\circ$C while stirring (10 min)
\item \textbf{Add 100 ml Palatant-Antioxidant premix (2 min):}
  
  \fbox{\begin{minipage}{0.9\textwidth}
  \textbf{CRITICAL STEP --- CANNOT BE SKIPPED}
  
  \textbf{Purpose:} Protects solids from oxidation during processing, provides palatability enhancement, and supplies primary Vitamin E
  
  \begin{itemize}
  \item \textbf{Premix A (standard antioxidant - RED STICKERs):} Use for Heart, Liver, Gizzard, Spleen, Kidney SKU
  \item \textbf{Premix B (enhanced antioxidant - BLUE STICKER):} Use for Sardine SKU ONLY 
  \end{itemize}
  
  $\checkmark$ Verify bottle label matches production schedule \\
  $\checkmark$ Check color coding on bottle \\
  $\checkmark$ \textbf{Obtain supervisor sign-off before adding}
  \end{minipage}}
  
\item Paddle mix for 2---3 minutes until fully dispersed (3 min)
\item \textbf{Total liquid: 4.20 kg} (broth + Palatant-Antioxidant)
\item Maintain steady temperature at 40~$^\circ$C
\item Skim foam if present 
\end{enumerate}

\textbf{QC Checkpoints:}
\begin{itemize}
\item pH 6.4---6.5 (recorded in batch log)
\item Temperature 40~$^\circ$C $\pm$ 2~$^\circ$C
\item Complete dispersion of Palatant (no oil separation)
\textbf{Log End Time in Timesheet:} \_\_\_\_\_
\textbf{Log Duration:} Target 30---35 minutes. If over this limit, halt and call supervisor. This step is essential to control pathogen load and flavor profile. 
\end{itemize}

% ================================================================
% STEP 1A/Step 2: SOLID PHASE PREPARATION
% ================================================================

\subsection*{Step 1A/ Step 2: Solid Phase Preparation and Steam Blanch}
\textbf{Perform this step in parallel to Step 1 with different operators and seperate work bench. 

\textbf{Estimated Time:} 35---40 minutes (operator active) \\
\textbf{Personnel Required:} 1 operator \\
\textbf{Equipment Required:} Steam blancher, collection vessel, scale

\begin{enumerate}[leftmargin=1.5em]
\textbf{Log Start Time in Timesheet:} \_\_\_\_\_
\textbf Ensure less than 5 minutes between Start Time in Step 1 so these two steps are in parallel to control pathogen load and flavor profile. 
\item Prepare raw solids according to SKU Bill of Materials (BOM) (10 min):
  \begin{itemize}
  \item Heart, Liver, Gizzard, Spleen, Kidney, Sardine $\rightarrow$ CatCore BOM
  \end{itemize}
\item Measure raw materials precisely: \textbf{6.50 kg total} for 100 pouches
\item \textbf{Note SKU color code on production log and obtain sign-off}
\item Steam blanch solid phase (5 min):
  \begin{itemize}
  \item Temperature: 90---95~$^\circ$C
  \item Time: 3---5 minutes
  \item Purpose: Partial cooking, microbial reduction, water absorption
  \end{itemize}
\item Collect and measure blanch water (5 min): \textbf{target 150 ml recovery}
\item \textbf{Carefully add blanch water back to Broth A (2 min):}
  \begin{itemize}
  \item Log exact volume recovered (typically 125---175 ml)
  \item Recovers water-soluble vitamins and taurine
  \item \textbf{Broth A now: 1.25 kg}
  \end{itemize}
\item \textbf{Blanched solids now: approximately 6.70 kg} (weight gained from water absorption)
\item Cool blanched solids slightly (to $\sim$60~$^\circ$C) for safe handling (10---15 min)
\textbf{Log End Time:} \_\_\_\_\_
\textbf{Log Duration:} \_\_\_\_\_. If more than 35---40 minutes halt and call Line Supervisor. 
\end{enumerate}

\textbf{QC Checkpoints:}
\begin{itemize}
\item Blanch water recovery volume recorded (250 ml, clear liquid acceptable)
\item Blanched solids weight recorded ($\sim$6.70 kg expected)
\item Visual inspection: no raw spots, uniform color
\end{itemize}

% ================================================================
% STEP 3: TWO-BROTH SYSTEM
% ================================================================

\subsection*{Step 3: Two-Broth System --- Split and Enrich}
\textbf{Estimated Time:} 15---20 minutes (operator active) \\
\textbf{Personnel Required:} 1 operator \\
\textbf{Equipment Required:} 2 labeled vessels, paddle mixer, calibrated pH meter

\begin{enumerate}[leftmargin=1.5em]
\textbf{Log Start Time in Timesheet:} \_\_\_\_\_
\textbf Ensure less than 5 minutes between End Time in Step 1 and Start Time in Step 3. 
\item Divide the 4.10 kg enriched broth into two vessels (5 min):
  \begin{itemize}
  \item \textbf{Broth A: 1.00 kg} $\rightarrow$ Label vessel ``BROTH A --- PREMIX''
  \item \textbf{Broth B: 3.10 kg} $\rightarrow$ Label vessel ``BROTH B --- GEL''
  \end{itemize}
\item Cover Broth B and maintain at steady 40~$^\circ$C (set aside for Step 6)
\end{enumerate}

\textbf{Enriching Broth A with vitamin/mineral premix:}
\begin{enumerate}[resume,leftmargin=1.5em]
\item Take the \textbf{color-coded premix packet} for today's SKU:
  \begin{itemize}
  \item RED sticker = CatPro (for CatCore SKUs)
  \end{itemize}
\item \textbf{VERIFY: Check production schedule matches sticker color}
\item \textbf{Obtain supervisor sign-off on premix selection}
\item Add \textbf{100 g premix} to Broth A while paddle mixing (2 min)
\item Mix steadily for 2---3 minutes until complete dispersion (3 min)
\item \textbf{Broth A now: 1.10 kg}
\item \textbf{pH Check \#2 (5 min):}
  \begin{itemize}
  \item Target: pH 6.4---6.5
  \item Adjust if needed
  \item \textbf{Obtain QC sign-off}
  \end{itemize}
\item Maintain Broth A at 40~$^\circ$C, covered
\end{enumerate}

\textbf{QC Checkpoints:}
\begin{itemize}
\item Correct premix color code verified
\item pH 6.4---6.5 (recorded in batch log)
\item Complete dispersion (no clumps or settling)
\textbf{Log End Time:} \_\_\_\_\_
\textbf{Log Duration:} \_\_\_\_\_. If more than 15---20 minutes halt and call Line Supervisor. 
\end{itemize}



% ================================================================
% STEP 4: MARINATION
% ================================================================

\subsection*{Step 4: Solid + Broth A Marination (Phase 1 Mix)}

\textbf{Estimated Time:} 45 minutes total (5 min active + 30 min waiting + 10 min active) \\
\textbf{Personnel Required:} 1 operator \\
\textbf{Equipment Required:} Large mixing vessel, paddle mixer, refrigeration unit, thermometer \\
\textbf{Parallel Tasks:} During 30-minute marination wait, operators can set up pouches for filling and prepare gel mixing equipment

\begin{enumerate}[leftmargin=1.5em]
\textbf{Log Start Time in Timesheet:} \_\_\_\_\_
\textbf Ensure less than 5 minutes between End Time in Step 2/ Step 3 and Start Time in Step 4. 
\item Transfer \textbf{6.70 kg blanched solids} to large mixing vessel (2 min).  
\item Add \textbf{1.25 kg Broth A} (with premix and blanch water) (1 min). 
\item Paddle mix gently for 1---2 minutes to coat evenly (2 min). 
\item Cover and marinate at \textbf{4~$^\circ$C for 30 minutes} (30 min wait)
  \begin{itemize}
  \item Purpose: Flavor absorption, premix distribution, meat tenderization
  \item \textbf{Note:} Operators can perform other prep tasks during this waiting period such as preparing the pouches for assembly line. 
  \end{itemize}
\item After 30 minutes, gently warm mixture to \textbf{60~$^\circ$C} (do not overcook) (10 min)
\item \textbf{Total Phase 1 mix: 8.05 kg} ready for filling
\textbf{Log End Time:} \_\_\_\_\_
\textbf{Log Duration:} \_\_\_\_\_. If more than 50 minutes halt and call Line Supervisor. 
\end{enumerate}

\textbf{QC Checkpoints:}
\begin{itemize}
\item Marination time recorded (exactly 30 minutes)
\item Final temperature: 60~$^\circ$C $\pm$ 5~$^\circ$C
\item Visual inspection: uniform coating, no dry spots
\end{itemize}

% ================================================================
% STEP 5: PHASE 1 FILL
% ================================================================

\subsection*{Step 5: Phase 1 Fill --- Solids + Broth A into Pouches}

\textbf{Estimated Time:} 50---60 minutes (operator active) \\
\textbf{Personnel Required:} 1---2 operators (2 operators recommended for faster filling) \\
\textbf{Equipment Required:} Filling line, calibrated scale (tared), 100+ pouches, filling ladle or dosing equipment

\begin{enumerate}[leftmargin=1.5em]
\item Set up filling line with \textbf{100 empty pouches + spares}. This can be done prior to Step 1. 
\item \textbf{Tare scale to zero with empty pouch on it} (pouch weight tared out during fill). This can be done prior to Step 1. 

\textbf{Log Start Time in Timesheet:} \_\_\_\_\_
\textbf Ensure less than 5 minutes between End Time in Step 4 and Start Time in Step 5. 
\item Fill each pouch with \textbf{80.5 g $\pm$ 0.5 g} of Solids + Broth A mixture (30---40 min)
  \begin{itemize}
  \item Manual filling: approximately 30 seconds per pouch
  \item With dosing equipment: approximately 15 seconds per pouch
  \end{itemize}
\item \textbf{In-line QC: Weigh every 10th pouch (5 min):}
  \begin{itemize}
  \item Record weight on production log
  \item Scale shows: 80---81 g (pouch tared to zero)
  \item \textbf{If any pouch deviates by $>$1 g, STOP LINE and report to supervisor}
  \end{itemize}
\item Continue until all 100 pouches filled with Phase 1 content
\item Keep pouches upright and stable on filling rack
\item \textbf{Do not seal yet} --- Phase 2 injection coming next
\textbf{Log End Time:} \_\_\_\_\_
\textbf{Log Duration:} \_\_\_\_\_. If more than 105 minutes halt and call Line Supervisor. 
\end{enumerate}

\textbf{QC Checkpoints:}
\begin{itemize}
\item Every 10th pouch weight recorded (80---81 g target)
\item Visual inspection: no spills, pouches upright
\item Total pouches filled recorded (should be 100)
\end{itemize}

% ================================================================
% STEP 6: BROTH B GEL PREPARATION AND INJECTION
% ================================================================

\clearpage

\subsection*{Step 6: Broth B Gel Preparation and Injection (Phase 2)}

\textbf{Estimated Time:} 30---35 minutes total \\
\textbf{Breakdown:}
\begin{itemize}
\item Gel preparation: 12 minutes (operator active)
\item Gel injection: 18---23 minutes (operator active)
\item QC checks: 3 minutes
\end{itemize}
\textbf{Personnel Required:} 2 operators (1 for mixing, 1 for injection) \\
\textbf{Equipment Required:} Paddle mixer, dosing gun or measured ladle, calibrated scale (NOT tared), viscosity reference standard, timer

\vspace{0.5em}
\fbox{\begin{minipage}{0.9\textwidth}
\textbf{TIMING CRITICAL:} Complete gel injection within 30 minutes of mixing to prevent premature gelation
\end{minipage}}

\vspace{0.5em}

\textbf{Gel preparation: Time Sensitive Step. Start Timer}
\begin{enumerate}[leftmargin=1.5em]
\textbf{Log Start Time in Timesheet:} \_\_\_\_\_
\textbf Ensure less than 5 minutes between End Time in Step 5 and Start Time in Step 6. 
\item Take \textbf{Broth B (3.20 kg)} at 40~$^\circ$C (verify temperature)
\item \textbf{Verify temperature: Must be 40---45~$^\circ$C, NEVER above 45~$^\circ$C}
\item Add \textbf{100 g Alginate-Ca premix} while paddle mixing continuously (2 min)
  \begin{itemize}
  \item Mix GENTLY to avoid excessive air incorporation
  \item Mix thoroughly to prevent lumps
  \end{itemize}
\item Mix for \textbf{2---3 minutes} to reach target viscosity (per vendor specification) (3 min)
\item \textbf{QC viscosity check (2 min):}
  \begin{itemize}
  \item Compare to reference standard
  \item \textbf{Reject batch if viscosity $>$10\% outside target range}
  \item Obtain QC sign-off
  \end{itemize}
\item \textbf{Immediate transfer to dosing line} (clock is ticking!) (2 min)
\item \textbf{Broth B + gel: 3.20 kg total}
\item \textbf{Start timer: Injection must complete within 20 minutes from this point. This is critical to ensure no premature gel set.}
\end{enumerate}

\textbf{Gel injection into pouches:}
\begin{enumerate}[resume,leftmargin=1.5em]
\item Using dosing gun or inline caliburated injector, inject \textbf{32.0 g $\pm$ 0.5 g} into each pouch (10 sec per pouch, total  1000sec  or 18---23 min)
  \begin{itemize}
  \item Avoid Manual injection. With a caliburated dosing gun/inline injector $/pm$0.5g and approximately 10 seconds per pouch.
  \end{itemize}
\item \textbf{In-line QC: Weigh every 10th pouch (scale NOT tared) (3 min):}
  \begin{itemize}
  \item Record TOTAL weight on production log
  \item Scale shows: 116---117 g (80.5 g Phase 1 + 32.00 g Phase 2 + 4 g pouch)
  \item \textbf{If consistently outside 115---118 g, STOP LINE and report}
  \end{itemize}
\item \textbf{Complete entire gel injection within 30 minutes from mixing}
\item \textbf{Target pre-retort food weight: 112.5 g per pouch}
\textbf{Log End Time:} \_\_\_\_\_
\textbf{Log Duration:} \_\_\_\_\_. If more than 25---30 minutes halt and call Line Supervisor.
\end{enumerate}

\textbf{QC Checkpoints:}
\begin{itemize}
\item Gel viscosity within specification (recorded)
\item Every 10th pouch total weight recorded (116---117 g target)
\item Injection completion time recorded (must be $<$30 min from gel mixing)
\end{itemize}

\textbf{Common Issues \& Solutions:}
\begin{itemize}
\item \textbf{Issue:} Gel sets too fast (viscosity $>$10\% above target)
  \begin{itemize}
  \item \textbf{Cause:} Temperature too high ($>$45~$^\circ$C) or mixing too long
  \item \textbf{Solution:} Discard batch, restart with cooler broth
  \end{itemize}
\item \textbf{Issue:} Injection taking $>$20 minutes
  \begin{itemize}
  \item \textbf{Cause:} Manual filling too slow
  \item \textbf{Solution:} Add second operator or switch to dosing gun
  \end{itemize}
\end{itemize}

% ================================================================
% STEP 7: SEAL AND RETORT
% ================================================================

\subsection*{Step 7: Seal and Retort}

\textbf{Estimated Time:} 120 minutes total (2 hours) \\
\textbf{Breakdown:}
\begin{itemize}
\item Sealing: 30--35 minutes (operator active)
\item Retort cycle: 75 minutes (Loging and monitoring f0, heat curves and penetration to center of the retort chamber)
\item Documentation: 5 minutes
\end{itemize}
\textbf{Personnel Required:} 2 operators (1 for sealing, 1 for retort loading) \\
\textbf{Equipment Required:} Nitrogen flush system, heat sealer, retort chamber, temperature monitoring equipment

\vspace{0.5em}
\textbf{Sealing (complete within 5 minutes of final fill):}
\begin{enumerate}[leftmargin=1.5em]
\textbf{Log Start Time in Timesheet:} \_\_\_\_\_
\textbf Ensure less than 5 minutes between End Time in Step 6 and Start Time in Step 7. 
\item Use in-line \textbf{nitrogen flush} for all pouches (prevents oxidation)
\item \textbf{Double-seal} each pouch (20--30 min):
  \begin{itemize}
  \item Seal width: 5 mm $\times$ 2 parallel bands
  \item Temperature: 180---190~$^\circ$C
  \item Dwell time: 2.5---3 seconds
  \item Approximately 15---20 seconds per pouch
  \end{itemize}
\item \textbf{Inspect every seal (10 min):}
  \begin{itemize}
  \item Reject if wrinkles, incomplete edges, or weak spots detected
  \item Set rejected pouches aside for rework or discard
  \end{itemize}
\item All accepted pouches proceed to retort
\end{enumerate}

\textbf{Retort process:}
\begin{enumerate}[resume,leftmargin=1.5em]
\item Load sealed pouches in retort racks (10 min):
  \begin{itemize}
  \item Maintain headspace $<$10 mm between pouches
  \item Water immersion or steam-air mode per equipment specification
  \end{itemize}
\item \textbf{Retort cycle parameters:}
  \begin{itemize}
  \item Come-up time: $\le$10 minutes
  \item Hold temperature: \textbf{121~$^\circ$C}
  \item Hold time: \textbf{45 minutes} (target $F_0 \ge 12$)
  \item Total cycle time: approximately 75 minutes
  \end{itemize}
\item \textbf{Temperature monitoring:}
  \begin{itemize}
  \item Record temperature trace continuously
  \item Monitor heat penetration in center of chamber
  \item \textbf{CRITICAL: Discard entire batch if any pouch $<$118~$^\circ$C for $>$5 minutes}
  \end{itemize}
\item Cool rapidly: \textbf{Below 40~$^\circ$C within 20 minutes}
\item \textbf{Record (5 min):}
  \begin{itemize}
  \item Time/temperature chart
  \item Any deviations from standard cycle
  \item Alternative validated $F_0$ curve if used
  \end{itemize}
\item \textbf{Obtain sign-offs from Line Supervisor AND QC Manager}
\textbf{Log End Time:} \_\_\_\_\_
\textbf{Log Duration:} \_\_\_\_\_. If more than 85---90 minutes halt and call Line Supervisor.
\end{enumerate}

\textbf{QC Checkpoints:}
\begin{itemize}
\item All seals inspected and approved
\item Retort temperature chart recorded and attached
\item Minimum $F_0$ value achieved (recorded)
\item Cooling time recorded (must be $<$20 min)
\end{itemize}

% ================================================================
% STEP 8: POST-RETORT QC
% ================================================================

\subsection*{Step 8: Post-Retort QC (within 1 hour of cooling)}

\textbf{Estimated Time:} 60 minutes total \\
\textbf{Breakdown:}
\begin{itemize}
\item Visual inspection: 15 minutes
\item Weight verification: 10 minutes
\item Destructive testing: 10 minutes
\item Labeling: 20 minutes
\item Documentation: 5 minutes
\end{itemize}
\textbf{Personnel Required:} 2 operators \\
\textbf{Equipment Required:} Calibrated scale, labels, photography equipment (for destructive test)

\vspace{0.5em}
\textbf{Visual inspection:}
\begin{enumerate}[leftmargin=1.5em]
\textbf{Log Start Time in Timesheet:} \_\_\_\_\_
\textbf Ensure less than 5 minutes between End Time in Step 7 and Start Time in Step 8. 
\item Inspect every pouch for (15 min):
  \begin{itemize}
  \item Seal integrity (no leaks, no delamination)
  \item Surface damage (punctures, tears)
  \item Deformation (excessive swelling, vacuum collapse)
  \end{itemize}
\item Set aside any questionable pouches for detailed inspection
\end{enumerate}

\textbf{Weight verification:}
\begin{enumerate}[resume,leftmargin=1.5em]
\item Randomly select \textbf{10 pouches per batch}
\item Weigh each pouch on calibrated scale (NOT tared, measuring total packaged weight) (10 min)
\item \textbf{QC Target: 105.5---106.5 g total packaged weight}
  \begin{itemize}
  \item This is: $\sim$101.5---102.5 g food + 4 g pouch
  \end{itemize}
\item Record all 10 weights on QC log
\item \textbf{Pass criteria: At least 9 out of 10 within specification}
\item \textbf{If $>$1 pouch fails: Inspect additional 20 pouches or reject batch}
\end{enumerate}

\textbf{Destructive testing (1 pouch per batch):}
\begin{enumerate}[resume,leftmargin=1.5em]
\item Select one pouch from middle of batch
\item Open carefully and evaluate (10 min):
  \begin{itemize}
  \item \textbf{Gel uniformity:} Firm structure, no mushiness
  \item \textbf{Syneresis:} $<$5\% free liquid on surface (minor weeping acceptable)
  \item \textbf{Odor:} Neutral to meaty, NO sour, rancid, or off-odors
  \item \textbf{Color:} Consistent with SKU specification (compare to reference photo)
  \item \textbf{Texture:} Gel holds shape when inverted, no excessive softness
  \end{itemize}
\item Record all observations with photos if needed
\item \textbf{Obtain QC sign-off on destructive test results}
\end{enumerate}

\textbf{Final steps:}
\begin{enumerate}[resume,leftmargin=1.5em]
\item Label all passed pouches with (20 min):
  \begin{itemize}
  \item Batch number
  \item Production date
  \item Best-by date (18 months from production)
  \item SKU code and color designation
  \end{itemize}
\item Store at \textbf{20 $\pm$ 2~$^\circ$C, RH $<$60\%}
\textbf{Log End Time:} \_\_\_\_\_
\textbf{Log Duration:} \_\_\_\_\_. If more than 65---70 minutes halt and call Line Supervisor.

\item \textbf{Retain 5 pouches per batch} for:
  \begin{itemize}
  \item Stability testing (shelf-life validation)
  \item Regulatory samples (if required)
  \item Customer complaint investigation (if needed)
  \end{itemize}
\item Complete all production logs and submit to QC for filing (5 min)
\end{enumerate}

\textbf{QC Checkpoints:}
\begin{itemize}
\item Visual inspection results recorded (pass/fail for each pouch)
\item Weight verification: 10 weights recorded, pass criteria met
\item Destructive test: photos attached, all parameters within spec
\item Labeling complete and verified
\item 5 retention samples properly stored and logged
\textbf{Log Final End Time:} \_\_\_\_\_
\textbf{Log Total Process Duration:} \_\_\_\_\_. 
\end{itemize}

% ================================================================
% PRODUCTION FLOW DIAGRAM
% ================================================================

\clearpage
\subsection*{Production Flow Diagram}

\begin{center}
\begin{tikzpicture}[
  node distance=1.2cm,
  every node/.style={font=\small},
  box/.style={rectangle, draw=black, thick, fill=blue!10, text width=7cm, align=center, minimum height=0.8cm},
  criticalbox/.style={rectangle, draw=red, very thick, fill=yellow!20, text width=7cm, align=center, minimum height=0.8cm},
  arrow/.style={->, >=stealth, thick}
]

\node[box] (start) {\textbf{START:} 4.10 kg Bone Broth (40 deg C), set aside 0.1kg for evaporation loss};

\node[criticalbox, below=of start] (palatant) {
  \textbf{ADD:} 0.10 Kg Liquid Palatant-Antioxidant A/B \\
  CRITICAL: Protects from oxidation \\
  A = Standard - RED STICKER | B = Sardine- BLUE STICKER (high-PUFA)
};

\node[box, below=of palatant] (total) {\textbf{TOTAL:} 4.10 kg Enriched Broth (40 deg C)};

\node[box, below=of total, xshift=-3.5cm] (brotha) {
  \textbf{BROTH A:} 1.00 kg \\
  (Premix carrier)
};

\node[box, below=of total, xshift=3.5cm] (brothb) {
  \textbf{BROTH B:} 3.10 kg \\
  (Gel system)
};

\node[box, below=of brotha] (premixa) {
  ADD: 0.10 kg CatPro \\
  ADD: 0.25 kg Blanch water \\
  \textbf{BROTH A: 1.35 kg}
};

\node[box, below=of brothb] (holdb) {
  HOLD at 40 deg C \\
  (covered, set aside)
};

\node[box, below=of premixa, yshift=-0.5cm] (solids) {
  6.50 kg Raw Solids \\
  Steam Blanch (90-95 deg C) \\
  6.70 kg Blanched Solids
};

\node[box, below=of solids] (phase1mix) {
  MIX: 6.70 kg Solids + 1.35 kg Broth A \\
  Marinate 30 min at 4 deg C \\
  \textbf{PHASE 1 MIX: 8.05 kg}
};

\node[box, below=of phase1mix] (fill1) {
  \textbf{PHASE 1 FILL:} 80.5 g per pouch \\
  (Scale tared, shows food only)
};

\node[box, below=of holdb, yshift=-6cm] (gel) {
  Broth B + 0.10 kg Alginate \\
  Mix 2-3 min \\
  \textbf{3.20 kg Gel System}
};

\node[box, below=of gel] (fill2) {
  \textbf{PHASE 2 FILL:} 32.0 g per pouch \\
  (Scale NOT tared, shows 116-117 g total)
};

\node[box, below=of fill2, xshift=-3.5cm, yshift=-0.5cm] (preretort) {
  \textbf{PRE-RETORT:} 112.5 g food/pouch \\
  Nitrogen flush + Double seal
};

\node[box, below=of preretort] (retort) {
  \textbf{RETORT:} 121 deg C x 45 min \\
  Moisture loss: -10 percent (-11.25 g)
};

\node[box, below=of retort] (postretort) {
  \textbf{POST-RETORT:} \\
  101 g food + 4 g pouch \\
  \textbf{QC: 104.5-105.5 g total}
};

\node[box, below=of postretort] (end) {
  \textbf{END:} 100 finished pouches \\
  Label, store (20+/-2 deg C, RH<60 percent)
};

% Arrows
\draw[arrow] (start) -- (palatant);
\draw[arrow] (palatant) -- (total);
\draw[arrow] (total) -| (brotha);
\draw[arrow] (total) -| (brothb);
\draw[arrow] (brotha) -- (premixa);
\draw[arrow] (brothb) -- (holdb);
\draw[arrow] (premixa) -- (solids);
\draw[arrow] (solids) -- (phase1mix);
\draw[arrow] (phase1mix) -- (fill1);
\draw[arrow] (holdb) -- (gel);
\draw[arrow] (gel) -- (fill2);
\draw[arrow] (fill1) |- (preretort);
\draw[arrow] (fill2) -| (preretort);
\draw[arrow] (preretort) -- (retort);
\draw[arrow] (retort) -- (postretort);
\draw[arrow] (postretort) -- (end);

\end{tikzpicture}
\end{center}

% ================================================================
% PRODUCTION TIMELINE SUMMARY
% ================================================================

\clearpage
\subsection*{Production Timeline Summary}

\begin{table}[h]
\centering
\caption{Complete Production Timeline for 100-Pouch Batch}
\label{tab:timeline}
\begin{tabular}{@{}llrrr@{}}
\toprule
\textbf{Step} & \textbf{Activity} & \textbf{Time (min)} & \textbf{Personnel} & \textbf{Can Parallel?} \\
\midrule
\textit{Pre-Production} & Broth rendering & 720 & 1 & N/A (overnight) \\
\midrule
1 & Liquid phase prep & 35---40 & 1 & No \\
2 & Broth split \& enrich & 15---20 & 1 & No \\
3 & Solid prep \& blanch & 35---40 & 1 & No \\
4 & Marination & 45 & 1 & Yes (30 min wait) \\
5 & Phase 1 fill & 50---60 & 1---2 & No \\
6 & Gel prep \& injection & 30---35 & 2 & Partially \\
7 & Seal \& retort & 120 & 2 & Log and Monitor \\
8 & Post-retort QC & 60 & 2 & No \\
\midrule
\multicolumn{2}{l}{\textbf{TOTAL PRODUCTION TIME}} & \textbf{390---450} & \textbf{2---3} & \\
\multicolumn{2}{l}{\textbf{(Approximately)}} & \textbf{6.5---7.5 hours} & & \\
\bottomrule
\end{tabular}
\end{table}

\vspace{1em}
\noindent\textbf{Notes on Timeline Optimization:}
\begin{itemize}
\item During Step 4 marination (30-minute wait), operators can set up pouches, prepare gel mixing equipment, and organize labeling materials
\item Steps 1---3 can be partially parallelized if 2 operators are available (one handles liquid prep, one handles solid prep)
\item Step 7 retort cycle is mostly waiting time; operators can begin cleaning and prep for next batch
\item With experienced operators and optimized workflow, total production time can approach 6.5 hours
\item First production run may take longer (7.5---8 hours) as operators familiarize themselves with procedures
\end{itemize}

% ================================================================
% CRITICAL CONTROL POINTS (CCP)
% ================================================================

\subsection*{Critical Control Points (CCP) Summary}

\begin{table}[h]
\centering
\caption{HACCP Critical Control Points for Retort Process}
\label{tab:ccp}
\begin{tabular}{@{}llll@{}}
\toprule
\textbf{CCP} & \textbf{Hazard} & \textbf{Critical Limit} & \textbf{Monitoring} \\
\midrule
Broth pH & Bacterial growth & pH 6.4---6.5 & Calibrated pH meter \\
 & & & (Step 1 \& 2) \\
\midrule
Palatant-Antioxidant & Lipid oxidation & 100 ml per batch & Visual verification \\
Addition & Vitamin E deficiency & Correct variant (A/B) & Supervisor sign-off \\
\midrule
Premix Addition & Micronutrient & 100 g per batch & Weight verification \\
 & deficiency & Correct type (CatCore) & Color code check \\
\midrule
Blanch Temperature & Pathogen survival & 90---95 deg C & Calibrated thermometer \\
 & & 3---5 min hold & Timer \\
\midrule
Fill Weight & Under/over filling & 80.5 +/- 0.5 g (Phase 1) & Scale check every \\
 & Nutritional imbalance & 32.0 +/- 0.5 g (Phase 2) & 10th pouch \\
\midrule
Gel Injection Time & Premature gelation & <30 min from mixing & Timer, visual check \\
\midrule
Seal Integrity & Contamination & No wrinkles, complete & 100 percent visual \\
 & Spoilage & edges, 2 bands & inspection \\
\midrule
Retort Temperature & Pathogen survival & 121 deg C, 45 min & Continuous chart \\
 & (C. botulinum) & F-value $\ge$ 12 & recorder \\
 & & Never <118 deg C & Temperature probe \\
\midrule
Cooling Time & Spore outgrowth & <40 deg C within 20 min & Calibrated thermometer \\
\midrule
Final Weight & Moisture loss & 104.5---105.5 g total & 10 random pouches \\
 & Quality control & (food + pouch) & per batch \\
\bottomrule
\end{tabular}
\end{table}

\vspace{1em}
\noindent\textbf{Corrective Actions:}
\begin{itemize}
\item If any CCP is outside critical limits, STOP production immediately
\item Notify Line Supervisor and QC Manager
\item Identify root cause before resuming
\item Affected product must be segregated and evaluated for:
  \begin{itemize}
  \item Rework (if safe and feasible)
  \item Downgrade (if food safety maintained but quality affected)
  \item Discard (if food safety cannot be assured)
  \end{itemize}
\item Document all deviations and corrective actions in batch log
\item Obtain QC Manager sign-off before resuming production
\end{itemize}

% ================================================================
% TROUBLESHOOTING GUIDE
% ================================================================

\subsection*{Troubleshooting Guide}

\begin{table}[h]
\centering
\caption{Common Issues and Solutions}
\label{tab:troubleshooting}
\begin{tabular}{@{}p{4cm}p{3.5cm}p{5cm}@{}}
\toprule
\textbf{Issue} & \textbf{Likely Cause} & \textbf{Solution} \\
\midrule
Broth pH outside range (6.4---6.5) & Ingredient variation, bacterial contamination & Adjust with food-grade citric acid (lower) or sodium bicarbonate (raise). If >0.3 units off, investigate source. \\
\midrule
Premix not fully dispersing in Broth A & Insufficient mixing time, temperature too low & Mix for additional 2 min. Ensure broth at 40 deg C. If clumping persists, sieve premix before adding. \\
\midrule
Blanched solids underweight (<6.65 kg) & Insufficient water absorption, over-trimming & Check blanch time (must be 3---5 min). Verify raw solid measurement. Adjust water addition if needed. \\
\midrule
Blanched solids overweight (>6.75 kg) & Excess water retention, under-blanching & Drain more thoroughly. Check blanch temperature (must be 90---95 deg C). \\
\midrule
Phase 1 fill inconsistent weights & Incomplete mixing, settling during filling & Gently stir mixture every 10 pouches. Ensure solids evenly distributed. Adjust technique if systematically high/low. \\
\midrule
Gel sets too quickly (<20 min working time) & Temperature >45 deg C, alginate concentration high & Check Broth B temperature before mixing. Use cooler broth (38---40 deg C). Work faster or add second operator. \\
\midrule
Gel too thin (syneresis >5 percent) & Temperature too low, alginate concentration low, insufficient mixing & Verify alginate addition (100 g). Ensure thorough mixing (2---3 min). Check Broth B temperature (40---42 deg C). \\
\midrule
Seal failures (wrinkles, incomplete) & Pouch overfilled, temperature incorrect, dwell time too short & Check fill weights (should be 112.5 g food). Verify sealer temp (180---190 deg C). Increase dwell to 3 sec. \\
\midrule
Retort F-value not achieved & Temperature probe malfunction, come-up time too long, insufficient hold & Recalibrate temperature probe. Check retort loading (must allow steam circulation). Extend hold time to 50 min if needed. \\
\midrule
Excessive swelling post-retort & Overfilling, incomplete seal, retort temperature too high & Reduce fill weight by 1---2 g. Inspect seals more carefully. Verify retort at 121 deg C (not higher). \\
\midrule
Vacuum collapse post-retort & Underfilling, cooling too rapid, headspace too large & Increase fill weight by 1---2 g. Slow cooling rate. Reduce nitrogen flush. \\
\midrule
Off-odors post-retort & Ingredient spoilage, incomplete sterilization, contamination & Check incoming ingredient quality. Verify retort cycle completed properly. Discard batch and investigate source. \\
\midrule
Final weight <104 g & Excessive moisture loss, underfilling & Check pre-retort weight (should be 112.5 g). Verify retort humidity. Increase fill by 1---2 g if systematic. \\
\midrule
Final weight >105 g & Insufficient moisture loss, overfilling & Check pre-retort weight (should be 112.5 g). Verify retort cycle. Reduce fill by 1---2 g if systematic. \\
\bottomrule
\end{tabular}
\end{table}

% ================================================================
% BATCH RECORD TEMPLATE
% ================================================================

\clearpage
\subsection*{Batch Record Template}

\noindent\textbf{Production Date:} \rule{3cm}{0.4pt} \hspace{2cm} \textbf{Batch Number:} \rule{3cm}{0.4pt}

\noindent\textbf{SKU:} \rule{4cm}{0.4pt} \hspace{2cm} \textbf{Target Quantity:} 100 pouches

\noindent\textbf{Line Supervisor:} \rule{4cm}{0.4pt} \hspace{1cm} \textbf{QC Inspector:} \rule{4cm}{0.4pt}

\vspace{1em}

\noindent\textbf{Step 1: Liquid Phase Preparation}

\begin{tabular}{@{}lp{3cm}p{3cm}p{3cm}@{}}
Bone broth weight: & \rule{2.5cm}{0.4pt} kg & Target: 4.10 kg & \\
pH (25 deg C): & \rule{2.5cm}{0.4pt} & Target: 6.4---6.5 & QC Sign: \rule{2cm}{0.4pt} \\
Palatant type: & $\Box$ A $\Box$ B & 100 ml added & Supervisor Sign: \rule{2cm}{0.4pt} \\
Final temp (40 deg C): & \rule{2.5cm}{0.4pt} deg C & Time: \rule{2cm}{0.4pt} & \\
\end{tabular}

\vspace{1em}

\noindent\textbf{Step 2: Two-Broth System}

\begin{tabular}{@{}lp{3cm}p{3cm}p{3cm}@{}}
Broth A weight: & \rule{2.5cm}{0.4pt} kg & Target: 1.00 kg & \\
Broth B weight: & \rule{2.5cm}{0.4pt} kg & Target: 3.20 kg & \\
Premix type: & $\Box$ CatPro & Color verified: $\Box$ & Supervisor Sign: \rule{2cm}{0.4pt} \\
Premix weight: & \rule{2.5cm}{0.4pt} g & Target: 100 g & \\
pH (Broth A): & \rule{2.5cm}{0.4pt} & Target: 6.4---6.5 & QC Sign: \rule{2cm}{0.4pt} \\
\end{tabular}

\vspace{1em}

\noindent\textbf{Step 3: Solid Phase Preparation}

\begin{tabular}{@{}lp{3cm}p{3cm}p{3cm}@{}}
Raw solids weight: & \rule{2.5cm}{0.4pt} kg & Target: 6.50 kg & \\
Blanch temp: & \rule{2.5cm}{0.4pt} deg C & Target: 90---95 deg C & \\
Blanch time: & \rule{2.5cm}{0.4pt} min & Target: 3---5 min & \\
Blanch water recovered: & \rule{2.5cm}{0.4pt} ml & Target: 150 ml & \\
Blanched solids weight: & \rule{2.5cm}{0.4pt} kg & Target: $\sim$6.70 kg & \\
\end{tabular}

\vspace{1em}

\noindent\textbf{Step 4: Marination}

\begin{tabular}{@{}lp{3cm}p{3cm}p{3cm}@{}}
Phase 1 mix weight: & \rule{2.5cm}{0.4pt} kg & Target: 7.95 kg & \\
Marination start: & \rule{2.5cm}{0.4pt} & Duration: 30 min & \\
Marination end: & \rule{2.5cm}{0.4pt} & Final temp: \rule{2cm}{0.4pt} deg C (target: 60 deg C) & \\
\end{tabular}

\vspace{1em}

\noindent\textbf{Step 5: Phase 1 Fill (every 10th pouch)}

\begin{tabular}{@{}lcccccccccc@{}}
Pouch \#: & 10 & 20 & 30 & 40 & 50 & 60 & 70 & 80 & 90 & 100 \\
Weight (g): & \rule{1cm}{0.4pt} & \rule{1cm}{0.4pt} & \rule{1cm}{0.4pt} & \rule{1cm}{0.4pt} & \rule{1cm}{0.4pt} & \rule{1cm}{0.4pt} & \rule{1cm}{0.4pt} & \rule{1cm}{0.4pt} & \rule{1cm}{0.4pt} & \rule{1cm}{0.4pt} \\
\end{tabular}

\noindent Target: 79---80 g (scale tared) \hspace{2cm} Operator Sign: \rule{3cm}{0.4pt}

\vspace{1em}

\noindent\textbf{Step 6: Gel Preparation \& Injection}

\begin{tabular}{@{}lp{3cm}p{3cm}p{3cm}@{}}
Broth B temp: & \rule{2.5cm}{0.4pt} deg C & Target: 40---45 deg C & \\
Alginate weight: & \rule{2.5cm}{0.4pt} g & Target: 100 g & \\
Mixing start time: & \rule{2.5cm}{0.4pt} & Viscosity OK: $\Box$ & QC Sign: \rule{2cm}{0.4pt} \\
Injection start: & \rule{2.5cm}{0.4pt} & Injection end: \rule{2.5cm}{0.4pt} & Duration: \rule{2cm}{0.4pt} min \\
\end{tabular}

\vspace{0.5em}

\noindent\textbf{Phase 2 Total Weights (every 10th pouch, scale NOT tared)}

\begin{tabular}{@{}lcccccccccc@{}}
Pouch \#: & 10 & 20 & 30 & 40 & 50 & 60 & 70 & 80 & 90 & 100 \\
Weight (g): & \rule{1cm}{0.4pt} & \rule{1cm}{0.4pt} & \rule{1cm}{0.4pt} & \rule{1cm}{0.4pt} & \rule{1cm}{0.4pt} & \rule{1cm}{0.4pt} & \rule{1cm}{0.4pt} & \rule{1cm}{0.4pt} & \rule{1cm}{0.4pt} & \rule{1cm}{0.4pt} \\
\end{tabular}

\noindent Target: 116---117 g (includes pouch) \hspace{2cm} Operator Sign: \rule{3cm}{0.4pt}

\vspace{1em}

\noindent\textbf{Step 7: Seal \& Retort}

\begin{tabular}{@{}lp{3cm}p{3cm}p{3cm}@{}}
Sealing complete: & \rule{2.5cm}{0.4pt} & Rejected pouches: \rule{2cm}{0.4pt} & \\
Retort load time: & \rule{2.5cm}{0.4pt} & Retort start: \rule{2.5cm}{0.4pt} & \\
Come-up time: & \rule{2.5cm}{0.4pt} min & Target: $\le$10 min & \\
Hold temp: & \rule{2.5cm}{0.4pt} deg C & Target: 121 deg C & \\
Hold time: & \rule{2.5cm}{0.4pt} min & Target: 45 min & \\
F-value: & \rule{2.5cm}{0.4pt} & Target: $\ge$12 & \\
Cooling end: & \rule{2.5cm}{0.4pt} & Temp: \rule{2cm}{0.4pt} deg C (target: <40 deg C) & \\
Chart attached: & $\Box$ Yes & Supervisor Sign: \rule{2.5cm}{0.4pt} & QC Sign: \rule{2.5cm}{0.4pt} \\
\end{tabular}

\vspace{1em}

\noindent\textbf{Step 8: Post-Retort QC}

\noindent\textit{Visual inspection:} Pouches rejected: \rule{2cm}{0.4pt} \hspace{1cm} Reason: \rule{6cm}{0.4pt}

\vspace{0.5em}

\noindent\textit{Weight verification (10 random pouches):}

\begin{tabular}{@{}lcccccccccc@{}}
Pouch \#: & \rule{1cm}{0.4pt} & \rule{1cm}{0.4pt} & \rule{1cm}{0.4pt} & \rule{1cm}{0.4pt} & \rule{1cm}{0.4pt} & \rule{1cm}{0.4pt} & \rule{1cm}{0.4pt} & \rule{1cm}{0.4pt} & \rule{1cm}{0.4pt} & \rule{1cm}{0.4pt} \\
Weight (g): & \rule{1cm}{0.4pt} & \rule{1cm}{0.4pt} & \rule{1cm}{0.4pt} & \rule{1cm}{0.4pt} & \rule{1cm}{0.4pt} & \rule{1cm}{0.4pt} & \rule{1cm}{0.4pt} & \rule{1cm}{0.4pt} & \rule{1cm}{0.4pt} & \rule{1cm}{0.4pt} \\
\end{tabular}

\noindent Target: 104.5---105.5 g \hspace{2cm} Pass: $\ge$9/10 within spec \hspace{2cm} Result: $\Box$ Pass $\Box$ Fail

\vspace{0.5em}

\noindent\textit{Destructive test (1 pouch):}

\begin{tabular}{@{}lp{8cm}@{}}
Gel uniformity: & $\Box$ Pass $\Box$ Fail \hspace{2cm} Notes: \rule{5cm}{0.4pt} \\
Syneresis: & $<$5\%: $\Box$ Yes $\Box$ No \hspace{2cm} Notes: \rule{5cm}{0.4pt} \\
Odor: & $\Box$ Pass $\Box$ Fail \hspace{2cm} Notes: \rule{5cm}{0.4pt} \\
Color: & $\Box$ Pass $\Box$ Fail \hspace{2cm} Notes: \rule{5cm}{0.4pt} \\
Texture: & $\Box$ Pass $\Box$ Fail \hspace{2cm} Notes: \rule{5cm}{0.4pt} \\
Photos attached: & $\Box$ Yes \hspace{3cm} QC Sign: \rule{3cm}{0.4pt} \\
\end{tabular}

\vspace{1em}

\noindent\textbf{Final Disposition:}

\begin{tabular}{@{}lp{8cm}@{}}
Pouches approved: & \rule{3cm}{0.4pt} / 100 \\
Pouches rejected: & \rule{3cm}{0.4pt} / 100 \hspace{2cm} Reason: \rule{5cm}{0.4pt} \\
Retention samples: & 5 pouches stored \hspace{2cm} Location: \rule{4cm}{0.4pt} \\
Batch status: & $\Box$ APPROVED $\Box$ REJECTED $\Box$ HOLD FOR INVESTIGATION \\
\end{tabular}

\vspace{1em}

\noindent\textbf{Final Sign-Offs:}

\begin{tabular}{@{}lp{5cm}p{5cm}@{}}
Line Supervisor: & \rule{4cm}{0.4pt} & Date: \rule{3cm}{0.4pt} \\
QC Manager: & \rule{4cm}{0.4pt} & Date: \rule{3cm}{0.4pt} \\
\end{tabular}

% ================================================================
% NOTES SECTION
% ================================================================

\clearpage
\subsection*{Notes}

\begin{itemize}[leftmargin=1.5em]
\item This SOP covers production flow for line operations only
\item Detailed QC sampling plans (ICP assays, vitamin assays, microbiological testing): See detailed SOP binder
\item Tolerance specifications and deviation handling procedures: See detailed SOP binder
\item Equipment calibration schedules and maintenance logs: See equipment maintenance records
\item All logs and sign-offs must be preserved per HACCP requirements for minimum 2 years
\item For pre-pilot kitchen test protocols and validation studies: See separate validation documents
\item Emergency contact information and escalation procedures: See safety manual
\end{itemize}

\vspace{1em}
\noindent\textbf{Revision History:}
\begin{itemize}[leftmargin=1.5em]
\item Rev H.3.1.0 (21 October 2025): Corrected mass balance, updated fill weights, added time logs, emphasized Palatant-Antioxidant critical step, added troubleshooting guide and batch record template
\item Rev H.3.0.0 (15 October 2025): Initial production SOP
\end{itemize}

\vspace{1em}
\noindent\textbf{Document Control:}
\begin{itemize}[leftmargin=1.5em]
\item Document Owner: Growlrr Foods Pvt Ltd, Quality Assurance Department
\item Review Frequency: Quarterly or after any significant process change
\item Next Review Date: 21 January 2026
\item Distribution: Line Supervisors, QC Managers, Production Operators
\item Controlled Copy: This is a controlled document. Do not copy without authorization.
\end{itemize}

% End of SOP 30