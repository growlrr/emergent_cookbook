\section*{Standard Operating Procedure (SOP): Oxidation Control and Freshness Assurance for PUFA-Rich Sardine Meat}

\textbf{Purpose:}  
To implement industry best practices for maintaining the freshness, oxidative stability, and shelf-life of high PUFA sardine meat used in pet food manufacturing.

\textbf{Scope:}  
Applies to raw material sourcing, handling, storage, and processing of sardine meat for pet food production.

\textbf{Responsibilities:}  
- Procurement team for ensuring supplier compliance with freshness criteria & maintaing optimum storage pre-production.
- Quality Assurance team for supplier certification (COA), storage certification,Iodine and peroxide monitoring pre- and post- retort.   
- Production team for in-line nitrogen flushing, correct use of premix-B for the Sardine Runs and controlled atmosphere packaging.  
- Testing Team for testing post retort and post-shelf life for peroxide values, rancidity, gel stability and palatability.  

\textbf{Procedure:}

\begin{enumerate}
  \item \textbf{Supplier Quality Assurance:}  
  \begin{itemize}
    \item Specify maximum allowable Peroxide Value (PV) of incoming sardine meat to be <= 5 meq/kg.  
    \item Require COA and testing protocols from suppliers confirming PV, Total Volatile Basic Nitrogen (TVB-N), and microbial levels within acceptable limits.
    \item Require certification for cold-chain, date of procurement and traceability. Reject frozen lots older than 1 week or without proper cold-chain cert.
  \end{itemize}

  \item \textbf{Raw Material Handling and Storage:}  
  \begin{itemize}
    \item Maintain cold chain at 0--4 ~\degree C from receiving dock through processing to prevent microbial growth and oxidation.  
    \item Minimize handling time; use rapid transport and refrigerated storage.
  \end{itemize}

  \item \textbf{Processing Controls:}  
  \begin{itemize}
    \item Implement rapid processing and blanching/freezing steps to arrest enzymatic and oxidative degradation.  
    \item Utilize nitrogen flushing or inert gas blanketing in processing equipment and packaging to minimize oxygen exposure.  
    \item Employ vacuum packaging or Modified Atmosphere Packaging (MAP) with low oxygen levels for finished products.
  \end{itemize}

  \item \textbf{Antioxidant Use:}  
  \begin{itemize}
    \item Use correct Liquid Antioxidant Premix-B for Sardine runs at 100ml per 10 Kg Production. 
    \item Verify label on the premix bottles and get sign-off from floor supervisor for Liquid Anti-Oxidant Premix-B for High PUFA Sardine runs.
  \end{itemize}

  \item \textbf{Physical Barriers:}  
  \begin{itemize}
    \item Apply alginate-based calcium-activated gel premix at optimum SOP for sardine sku to reduce oxygen permeability and moisture exchange.  
    \item Ensure gel coatings are uniform and stable without synerisis.
  \end{itemize}

  \item \textbf{Quality Monitoring and Documentation:}  
  \begin{itemize}
    \item Routinely measure peroxide value, TBARS (thiobarbituric acid reactive substances), Iodine and sensory indicators during storage.  
    \item Maintain detailed batch records including raw material PV, processing dates, antioxidant lot numbers, and packaging conditions.
  \end{itemize}
\end{enumerate}

\textbf{Safety and Compliance:}  
All procedures must comply with FDA, AAFCO, and other local regulatory requirements governing pet food raw materials and additives.

\vspace{0.5em}


